\documentclass[letterpaper,12pt]{article}
\usepackage{array}
\usepackage{threeparttable}
\usepackage{geometry}
\geometry{letterpaper,tmargin=1in,bmargin=1in,lmargin=1.25in,rmargin=1.25in}
\usepackage{fancyhdr,lastpage}
\pagestyle{fancy}
\lhead{}
\chead{}
\rhead{}
\lfoot{}
\cfoot{}
\rfoot{\footnotesize\textsl{Page \thepage\ of \pageref{LastPage}}}
\renewcommand\headrulewidth{0pt}
\renewcommand\footrulewidth{0pt}
\usepackage[format=hang,font=normalsize,labelfont=bf]{caption}
\usepackage{listings}
\lstset{frame=single,
  language=Python,
  showstringspaces=false,
  columns=flexible,
  basicstyle={\small\ttfamily},
  numbers=none,
  breaklines=true,
  breakatwhitespace=true
  tabsize=3
}
\usepackage{amsmath}
\usepackage{amssymb}
\usepackage{amsthm}
\usepackage{harvard}
\usepackage{setspace}
\usepackage{float,color}
\usepackage[pdftex]{graphicx}
\usepackage{hyperref}
\hypersetup{colorlinks,linkcolor=red,urlcolor=blue}
\theoremstyle{definition}
\newtheorem{theorem}{Theorem}
\newtheorem{acknowledgement}[theorem]{Acknowledgement}
\newtheorem{algorithm}[theorem]{Algorithm}
\newtheorem{axiom}[theorem]{Axiom}
\newtheorem{case}[theorem]{Case}
\newtheorem{claim}[theorem]{Claim}
\newtheorem{conclusion}[theorem]{Conclusion}
\newtheorem{condition}[theorem]{Condition}
\newtheorem{conjecture}[theorem]{Conjecture}
\newtheorem{corollary}[theorem]{Corollary}
\newtheorem{criterion}[theorem]{Criterion}
\newtheorem{definition}[theorem]{Definition}
\newtheorem{derivation}{Derivation} % Number derivations on their own
\newtheorem{example}[theorem]{Example}
\newtheorem{exercise}[theorem]{Exercise}
\newtheorem{lemma}[theorem]{Lemma}
\newtheorem{notation}[theorem]{Notation}
\newtheorem{problem}[theorem]{Problem}
\newtheorem{proposition}{Proposition} % Number propositions on their own
\newtheorem{remark}[theorem]{Remark}
\newtheorem{solution}[theorem]{Solution}
\newtheorem{summary}[theorem]{Summary}
%\numberwithin{equation}{section}
\bibliographystyle{aer}
\newcommand\ve{\varepsilon}
\newcommand\boldline{\arrayrulewidth{1pt}\hline}

\begin{document}

\begin{flushleft}
  \textbf{\large{Problem Set \#1}} \\
  MACS 30100, Dr. Evans\\
  Shuting Chen 
\end{flushleft}

\vspace{5mm}

\noindent\textbf{Problem 1. Classify a model from a journal.}
\vspace{2mm}

\noindent\textbf{Parts (a) - (c).} 
\vspace{1mm}

\noindent\text{Detailed citation of the article:} 

\noindent Cornelissen, Thomas, Christian Dustmann, and Uta Sch\"{o}nberg. 2017. ``Peer Effects in the Workplace." \textit{American Economic Review}, 107(2): 425-56. 
\vspace{2mm}

\noindent Description of the theoretical model from the selected article: 

The authors develop a principal-agent model of unobserved worker effect in which peer-induced productivity effects arise through social pressure and knowledge spillover and translate into peer-related wage effects. 

Specifically, they consider a firm of \textit{N} workers and the production function for worker \textit{i} is defined as
\begin{equation} \label{1}
f_i = y_i + \varepsilon_i = a_i + e_i\,(1 + \lambda^K \bar{a}_{\sim i}) + \varepsilon_i, 
\end{equation}
where $y_i$ is worker $i$'s productive capacity, which depends on individual ability $a_i$, individual effort $e_i$, and average peer ability (excluding worker $i$) $\bar{a}_{\sim i}$. Besides, the knowledge spillover is captured by $\lambda^K$. 

Apart from modelling individual output $f_i$, the authors construct the following disutility function by including cost of effort and peer pressure 
\begin{equation} \label{2}
    c_i = C(e_i) + P(e_i, \bar{f}_{\sim i}) = k\,e_i^2 + \lambda^P (m-e_i)\bar{f}_{\sim i}
\end{equation}
where $\lambda^P$ and $m$ can be treated as both the ``strength" and the ``pain" from peer pressure and $\bar{f}_{\sim i}$ is the average peer output (excluding worker $i$). 
\vspace{1mm}

\noindent\textbf{Part (d).} \\
Exogenous variables: $a_i$, $e_i$, $\bar{a}_{\sim i}$ and $\bar{f}_{\sim i}$\\
Endogenous variables: $f_i$ and $c_i$ 
\vspace{1mm}

\noindent\textbf{Part (e).} \\
The model is static since it does not have any time index for each variable. Moreover, due to the nonlinear term $e_i\,\bar{a}_{\sim i}$ in function (\ref{1}) and quadratic cost of effort in function (\ref{2}), the model is nonlinear. Besides, this model is stochastic since it has $\varepsilon_i$, which is random productivity shock that is beyond the workers' control. 
\vspace{1mm}

\noindent\textbf{Part (f).} \\
Presumably, one could include a variable representing average peer effort(excluding worker $i$) in the production function for each worker (i.e. $\bar{e}_{\sim i}$). This should be reasonable since workers may increase their production when they perceive that others work harder than themselves. 


\newpage

\noindent\textbf{Problem 2. Make your own model.} 
\vspace{3mm}

\noindent\textbf{Model.} \\
To model whether someone decides to get married, which is a binary response variable, I am going to use the following logistic regression model
\begin{equation}
    P(y_i = 1|\textbf{x}) = G(\beta_0 + \beta_1r_i + \beta_2w_i + \beta_3a_i + \beta_4pm_i + \beta_5h_i + \beta_6eth_i)
\label{3}
\end{equation}
where $G(\cdot)$ is the cumulative distribution function (CDF) for a standard logistic random variable, $r_i$ is a dummy variable indicating whether individual $i$ is in a stable relationship, $w_i$ represents $i$'s total wealth including income, saving, real estate, etc., $a_i$ represents $i$'s age in numerical numbers, $pm_i$ illustrates the parents marital status (divorced or not) for individual $i$, $h_i$ records whether $i$ is in a stable health status without suffering from severe disease within two years, and $eth_i$ is ethnicity. \vspace{2mm}

\noindent\textbf{Data Generating Process.} \\
We could generate data of $y_i$ as follows
\begin{equation}
 y_i=
    \begin{cases}
      1, & \text{if $y_{i}^{\ast}> 0$ (get married)}\\
      0, & \text{otherwise (not get married)} 
    \end{cases}
    \label{4}
\end{equation}
where $y_{i}^{\ast}$ is a latent variable specified as 
\begin{equation}
 y_{i}^{\ast} = \beta_0 + \beta_1r_i + \beta_2w_i + \beta_3a_i + \beta_4pm_i + \beta_5h_i + \beta_6eth_i + \varepsilon_i, 
 \label{5}
\end{equation}
and $\varepsilon_i$ follows the standard logistic distribution. \vspace{2mm}

\noindent\textbf{Key Factors.}\\
Among these six chosen variables, I think the first three variables, which are whether individuals are in a stable relationship, individual's wealth condition, and age, have most important impact on someone's decision of marriage. Firstly, being in a relationship is a prerequisite for considering marriage. However, most responsible people would not consider marriage until they have established a healthy and stable relationship with their partner. Thus, whether or not being in a stable relationship would be a key factor when someone considers whether to get married. Besides, people would commonly take their wealth level into account before marriage. Marriage actually means that two people form a new family, which involves setting up a new home and presumably raising the next generation in the near future. All of these require a solid source of wealth. In addition, someone would like to get married because they are at marriageable age. Within a certain age range, people are under more pressure to get married when they become older. 
\vspace{2mm}

\noindent\textbf{Preliminary Test.}\\
A preliminary test could be conducted by using data collected from a relatively small scale survey designed for the desired variables. After collecting data, we could estimate the logit model (\ref{3}) by maximum likelihood estimation and implement statistical tests such as the likelihood ratio test to see the joint significance of chosen factors. Moreover, we could measure the goodness-of-fit of the model by pseudo $R^2$, $$\tilde{R}^2 = 1 - \frac{L_{ur}}{L_0},$$ where $L_{ur}$ is the log-likelihood for the unrestricted model and $L_0$ is the log-likelihood for the model estimated with only an intercept term. 

\end{document}


